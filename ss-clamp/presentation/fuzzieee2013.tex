% Copyright 2013 by Filipe Martins <fmm@cin.ufpe.br>
% This file can be redistributed and/or modified under the terms of the GNU Public License, version 2
%
% wfb 2013 conference
% talk lenght: about 15 minutes

\documentclass{beamer}

\mode<presentation>
{
	\usetheme{Warsaw}
}

\usepackage{verbatim}

\usepackage[english]{babel}

\usepackage{amsmath}

\usepackage{pgfplots}
\usepackage{cite,graphics,graphicx}

\usepackage{float}
\usepackage[tight,footnotesize]{subfigure}

% set maximum depth to 9 for itemize
\usepackage{enumitem}
\setlistdepth{9}
\setlist[itemize,1]{label=$\bullet$}
\setlist[itemize,2]{label=$\bullet$}
\setlist[itemize,3]{label=$\bullet$}
\setlist[itemize,4]{label=$\bullet$}
\setlist[itemize,5]{label=$\bullet$}
\setlist[itemize,6]{label=$\bullet$}
\setlist[itemize,7]{label=$\bullet$}
\setlist[itemize,8]{label=$\bullet$}
\setlist[itemize,9]{label=$\bullet$}
\renewlist{itemize}{itemize}{9}

\title[Semi-supervised fuzzy c-medoids clustering algorithm]{Semi-supervised fuzzy c-medoids clustering algorithm with multiple prototype representation}

\author{
	Filipe M. de Melo\inst{1} \and Francisco de A.T. de Carvalho\inst{1}
}

\institute{
	Center of Informatics\\
	Federal University of Pernambuco
}

\date[fuzzIEEE 2013]{2013 IEEE International Conference on Fuzzy Systems}

\subject{Machine Learning}

\AtBeginSubsection[]
{
  \begin{frame}<beamer>{Outline}
		\tableofcontents[currentsection,currentsubsection]
	\end{frame}
}

% begin
\begin{document}

\begin{frame}
	\titlepage
\end{frame}

\begin{frame}{Outline}
	\tableofcontents
\end{frame}

% introduction goes here
\section{Introduction}
\subsection{Motivation}
\subsection{Semi-supervised Clustering}

% ss-clamp goes here
\section{Proposed Algorithm}

%
\subsection{Details}
%% Parameters
%% Objective Function

% described the parameters used in the clustering process
\begin{frame}{Parameters}
\begin{itemize}
\item{$E=\{e_{1},\ldots,e_{N}\}$}
\item{$D_{j} = [d_{j}(e_{i},e_{l})]\,(j=1,\ldots,T)$}
\end{itemize}
\end{frame}

\begin{frame}{Objective Function}
% objective function
\begin{eqnarray*}
  % adequacy criterion to produce a consensus clustering
  J&{}=
  \pause
  {}&\sum_{i=1}^{N}\sum_{k=1}^{C}(u_{ik})^{2}\sum_{j=1}^{T}\lambda_{kj}\sum_{e \in G_{k}}d_{j}(e_{i},e)\\
  \pause
  % pairwise constraints
  &&{+}\:\alpha\left (\sum_{(l,m)\in\mathcal{M}}\sum_{r=1}^{C}\sum_{\substack{s=1 \\{s}\neq{r}}}^{C}u_{lr}u_{ms}+\sum_{(l,m)\in\mathcal{C}}\sum_{r=1}^{C}u_{lr}u_{mr}\right )
\end{eqnarray*}
  \pause
subject to
% membership restrictions (fuzzy)
\begin{displaymath}
  u_{ik} \geq 0\qquad\sum_{k=1}^{C}u_{ik} = 1\qquad\forall i
\end{displaymath}
% relevance weight restrictions
\begin{displaymath}
  \lambda_{kj} > 0\qquad\prod_{j=1}^{T}\lambda_{kj} = 1\qquad\forall k.
\end{displaymath}
\end{frame}

% experiments goes here

% end
\end{document}
